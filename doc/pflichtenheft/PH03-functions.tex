\section{Product Functions}

%\comment{functional Requirements}

\subsection{General Features}
\begin{description}
  \item[F1.1] The app shall use the Bluetooth adapter of the smartphone and the Android.bluetooth library to connect to a TI SimpleLink SensorTag device.
%  \item[F1.2] The app shall provide a live data view of the sensor feedback in human readable form.
  \item[F1.2] The app shall be able the record data from a connected sensor.
  \item[F1.3] The app shall display recorded data in a VR-World
  \item[F1.4] The app shall track connected TI SimpleLink SensorTag devices by tracking the position of the cellphone.
\end{description}

\subsection{VR-World}
  The VR-Mode is a 3D view of the world. When entering VR-Mode the user will see a fullscreen 3D world and by pressing the button in the lower right corner he can enter the  stereoscopic view of the World.
  The VR-World is a 3D representation of a real series of rooms.
  \begin{description}
    \item[F2.1] The VR-World shall be able to be viewed inside a web browser and from within the app.
    \item[F2.2] While viewing the VR-World the user shall be able to look around using the gyro sensor of his phone to pan the camera around.
    \item[F2.3] While the app is not in stereoscopic 3D mode the user shall be able to click and drag to pan the camera around.
    \item[F2.4] The app shall be able to move the camera inside the VR-World by using a bluetooth controller.
    \item[F2.5] The data fechted from the sensors shall be displayed inside the VR-World.
    \item[F2.6] When in VR-Mode, the app shall be in fullscreen mode.
    \item[F2.7] The app shall exit the VR-Mode if the user is pressing the ``x'' in the top right corner of the screen.
    \item[F2.8] The app shall be able to switch between stereoscopic 3D and normal 3D mode.
    \item[F2.9] The app shall be able to switch from fullscreen VR-Mode to stereoscopic by pressing the button in the lower right corner or by pressing the A-Button on his controller.
    \item[F2.10] The app shall be able to exit by pressing the back button on his device or by touching the back button in the top left corner.
    \item[F2.11] The app shall be able to switch to the settings screen, while it's in normal 3D mode.
    \item[F2.12] The app shall be able to switch rooms if the user pushes the B-Button on his controller.
    \item[F2.13] The app shall visualize the position of stored data from the TI SimpleLink SensorTag device.
    \item[F2.14] The app shall visualize the given data by the Ti SimpleLink SensorTag, by displaying a point approximately at the sensors stored location, with a number for the value of the the data.
    \item[F2.15] The app shall visualize the data by spanning a mesh over all recorded points from the sensor, while the height is the value of the given data.
    \item[F2.16] The User shall be able to switch between the two representation by pressing the X-Button on his controller.
  \end{description}



  \subsection{Settings}
    The user can set the following options:

    \begin{description}
      \item[F3.1] The app shall be configuarable so that the user may choose wich data shall be displayed in the VR-World (temperature, etc.).
      \item[F3.2] The app shall list the connected devices and a short info about the current setting and state of the TI SimpleLink SensorTag device.
      \item[F3.4] The app shall list the results of a Bluetooth scan and present an user interface for controlling the connection of TI SimpleLink SensorTag devices.
    \end{description}
