\section{Proposed Architecture}

\subsection{Overview}

\subsubsection{Services}
From \href{https://developer.android.com/guide/components/services.html}{AndroidDoc}: \\
``A Service is an application component that can perform long-running operations in the background, and it does not provide a user interface''.
\begin{itemize}
  \item \textbf{SensorTagBluetoothReceiverService:} Uses the android.bluetooth and especially the android.bluetooth.le libraries to fetch the sensor data from the TI CC2650. \\
  \item \textbf{SensorTagTrackingService:} Handles the tracking of the TI SensorTag devices.
  \item \textbf{nodejsService:} uses \href{https://github.com/paddybyers/anode}{Paddy Byer's anode framework} to start a node.js server on the localhost
\end{itemize}


\subsubsection{Activities}
From \href{https://developer.android.com/guide/components/services.html}{AndroidDoc}: \\
``They (Activities) serve as the entry point for a user's interaction with an app, and are also central to how a user navigates within an app (as with the Back button) or between apps (as with the Recents button)''. \newline
\begin{itemize}
  \item \textbf{MainActivity} Provides the main startup screen as the main entry point.
  \item \textbf{VRViewActivity} shall provide the WebVR view using the android.webkit library (especially .webview).
  \item \textbf{LiveDataActivity} shall provide a view of the sensor data in human readable form.
  \item \textbf{TISettingsActivity:} Settings screen containing scanning and connecting, connected devices and device settings fragments.
  \begin{itemize}
    \item \textbf{ScanningConnectingFragment} shall show the scanning results, delivered by the SensorTagBluetoothReceiverService and controll to which device to connect to or disconnect.
    \item \textbf{ConnectedDevicesFragment} shall show a list of all connected devices and a short info about the current setting and state of the TI SimpleLink SensorTag device.
    \item \textbf{ConnectedDevicesSettingsFragment} shall implement the configuration of the app features of the sensor.
  \end{itemize}
\end{itemize}

\subsubsection{Additional Classes}
\begin{itemize}
  \item \textbf{GATT Profiles} (for each sensor one)
  \item \textbf{GATT Sensor Service UUIDs}
  \item \textbf{Parser Functions} because the BLE protocol implemented in the TI CC2650 delivers raw sensor output
\end{itemize}
